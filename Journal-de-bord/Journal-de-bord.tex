\documentclass[12pt]{report}
\usepackage{fontspec}
\usepackage{xunicode}
\usepackage{polyglossia}
\setmainlanguage{french}
\usepackage{csquotes}

\usepackage{listings}

\usepackage{soul}
\usepackage{color}

\usepackage{enumitem}
\usepackage{setspace}

\usepackage{lscape}
\usepackage{graphicx}
\usepackage{float}

\title{Projet Jeux Olympiques - Journal de bord}
\date{Janvier 2024}
\author{T. Burnel, N. Grim, M. Griveau, M. Mechentel}

\begin{document}
	\setstretch{1.5}
	\maketitle
	
	\section{Introduction}
	
	Nous sommes une équipe de datajournalistes chargée d'étuder l'existence ou non de liens entre le succès d'un pays aux Jeux Olympiques et sa richesse.
	
	\section{Jeux de données}	
	
	\section{Traitement des données}
	
		\subsection{Objectif du traitement}
		
		\subsection{Chaîne de traitement}
		
		Pensez à parler aussi de vos échecs, des enrichissements ou croisements qui n’ont pas marché… et de ce que vous avez appris.
		
			\subsubsection{Requête SPARQL}
			
			Pour enrichir nos données \emph{via} Wikidata, notre objectif était de mettre au point une requête SPARQL retournant le nombre d'habitants de tous les pays du monde sur trente ans (1993 - 2003).
			
			\begin{verbatim}
				SELECT ?paysLabel ?population ?date
				WHERE
				{
					?pays wdt:P31 wd:Q6256.
					?pays p:P1082 ?populationStatement.
					?populationStatement ps:P1082 ?population.
					?populationStatement pq:P585 ?date.
					
					FILTER(YEAR(?date) >= (YEAR(NOW()) - 30)).
					
					SERVICE wikibase:label { bd:serviceParam wikibase:language "[AUTO_LANGUAGE],fr". }
				}
				ORDER BY ?paysLabel ?date
			\end{verbatim}
		
			Être en mesure de requêter la liste de l'ensemble des pays a été la première étape de la construction de notre requête. Le premier triplet utilise la variable inconnue \textbf{?pays} dont l'objet est \textbf{country} (wd:Q6256) :
			
			\begin{verbatim}
				?pays wdt:P31 wd:Q6256.
			\end{verbatim}
			
			La deuxième partie de la requête doit retourner la liste du nombre d'habitants de chaque pays en prenant en compte une dimension chronologique. La complexité de cette demande correspond à un parcours de graph en quatre temps -- et non pas en trois.
			
			Nous avons donc créé une nouvelle variable, \textbf{populationStatement?}, contenant toutes les propriétés utilisées dans la classe \textbf{population} grâce au préfixe \textbf{p}\footnote{Pour ce faire, nous avons consulté la liste suivante de préfixes Wikidata : }.

	\section{Visualisation des données}
	
		\subsection{Présentation des visualisations}
		
		\subsection{Analyse}
		
		Qu'apprend-on en regardant les visualisations ? Quels sont les biais ?


	\tableofcontents

\end{document}