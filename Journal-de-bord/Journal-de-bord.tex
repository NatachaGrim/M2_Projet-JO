\documentclass[12pt]{report}
\usepackage{fontspec}
\usepackage{xunicode}
\usepackage{polyglossia}
\setmainlanguage{french}
\usepackage{csquotes}

\usepackage{listings}

\usepackage{soul}
\usepackage{color}

\usepackage{enumitem}
\usepackage{setspace}

\usepackage{lscape}
\usepackage{graphicx}
\usepackage{float}

\title{Projet Jeux Olympiques - Journal de bord}
\date{Janvier 2024}
\author{T. Burnel, N. Grim, M. Griveau, M. Mechentel}

\begin{document}
	\setstretch{1.5}
	\maketitle
	
	\section{Introduction}
	
	Nous sommes une équipe de datajournalistes chargée d'étuder l'existence ou non de liens entre le succès d'un pays aux Jeux Olympiques et sa richesse.
	
	\section{Jeux de données}	
	
	\section{Traitement des données}
	
		\subsection{Objectif du traitement}
		
		\subsection{Chaîne de traitement}
		
			\subsubsection{Requête SPARQL}
		
		Pensez à parler aussi de vos échecs, des enrichissements ou croisements qui n’ont pas marché… et de ce que vous avez appris.
	
	\section{Visualisation des données}
	
		\subsection{Présentation des visualisations}
		
		\subsection{Analyse}
		
		Qu'apprend-on en regardant les visualisations ? Quels sont les biais ?


	\tableofcontents

\end{document}